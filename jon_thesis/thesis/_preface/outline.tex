\chapter*{Thesis Outline\markboth{Thesis Outline}{Thesis Outline}}

The body of this thesis describes the creation of a novel method for building sections
of protein structure -- \arcus\ -- intended primarily as an augmentation to the more general technique of comparative modelling. The primary role of \arcus\ 
is in modelling surface loops, where no information can
be obtained from homologous  template structure(s).

In order to accomplish this, it is important to begin in chapter \ref{chapter:protein_structure}, by discussing the intricacies of protein structure; specifically common classifications and structural motifs that any viable modelling method needs
to cover. From this point, chapter \ref{chapter:protein_modelling} attempts to orientate the reader
 with respect to current modelling methodologies and illustrate the role that \arcus\
 has to play within the field. 
 
Following this, chapter \ref{chapter:software} describes the development
of two entirely new, highly flexible, software frameworks 
with complementary functionality. The first framework -- \pd\ -- aimed at providing comprehensive bio-simulation capabilities,  forms a foundation for \arcus. Using the abilities of the second framework, chapter \ref{chapter:database} describes the development and characterisation of a representative sub-set
of the PDB, named \thothdb. This representative database is then used as a standardised
test-set for calibration and testing  throughout
the remainder of the thesis.

With solid groundwork in place, chapter \ref{chapter:reduced_rep} discusses the continuing development
of a reduced, but highly representative, low-resolution protein model and its
application to the problem of surface loop modelling. Chapter \ref{chapter:prearcus} then discusses the refinement of these low-resolution models into geometrically-correct low-energy models. Together these comprise the immature algorithm \prearcus. An initial performance-test was performed via the submission of predictions to  \casp-7 -- the protein structure prediction
competition. Initial
qualitative findings from this are presented in chapter \ref{chapter:casp}. 

Next, chapter \ref{chapter:methods} presents a unique and comprehensive overview of loop modelling methodology. The performance of six recent loop modelling methodologies is critically compared on the most comprehensive loop test-set yet analysed in the literature. Finally, improvements and algorithmic refinements
made to \arcus\ following the \casp\ competition and method-comparison are presented in chapter \ref{chapter:arcus}.
Brief concluding remarks are made
and further directions discussed in chapter \ref{chapter:discussion}.

